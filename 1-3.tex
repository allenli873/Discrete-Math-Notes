\documentclass[a4paper]{report}

\usepackage[utf8]{inputenc}
\usepackage[T1]{fontenc}
\usepackage{textcomp}
\usepackage{amsmath, amssymb}
%\usepackage{amsthm}
\usepackage{xcolor}
\usepackage{tcolorbox}
\tcbuselibrary{theorems}
\newtcbtheorem
  []% init options
  {definition}% name
  {Definition}% title
  {%
    colback=green!5,
    colframe=green!35!black,
    fonttitle=\bfseries,
  }% options
  {def}% prefix

\newtcbtheorem
  []% init options
  {theorem}% name
  {Theorem}% title
  {%
    colback=red!5,
    colframe=red!35!black,
    fonttitle=\bfseries,
  }% options
  {thrm}% prefix

\title{Discrete Math Notes}
\author{Allen Li}
% figure support
\usepackage{import}
\usepackage{xifthen}
\pdfminorversion=7
\usepackage{pdfpages}
\usepackage{transparent}
\newcommand{\incfig}[1]{%
    \def\svgwidth{\columnwidth}
    \import{./figures/}{#1.pdf_tex}
}
\DeclareMathSymbol{\lsim}{\mathord}{symbols}{"18}
\usepackage{hyperref}
\hypersetup{
    colorlinks,
    citecolor=black,
    filecolor=black,
    linkcolor=black,
    urlcolor=black
}

\begin{document}
\maketitle

\newpage
\tableofcontents{}
\newpage

\chapter{Speaking Mathematically}

\section{Statements}

\subsection{The Three Types of Statements}

\begin{enumerate}
    \item Universal Statement: Statement that applies "for all"
        \begin{itemize}
            \item Ex: For all real numbers $x$, $x^2 \ge 0$
            \item Key Words: "For All"
        \end{itemize}
    \item Conditional Statement: If one thing is true then some other thing must be true
        \begin{itemize}
            \item Key Words: "If, then"
        \end{itemize}
    \item Existential Statement: Statement that says there is at least one thing for which property is true
        \begin{itemize}
            \item There exists an even prime number.
            \item Key Words: "There exists"
        \end{itemize}
    We use these three statements in order to classify different statements.
    
\end{enumerate}

\subsection{Universal Conditional Statements}

Ex: For all real numbers, if $|x| > 1$, then  $x^2 > 1$.

This statement is universal because it has the "for all", and conditional because it has "if".

\subsection{Universal Existential Statement}

Ex: For all integers $n$, there exists another integer $m$, such that $n < m$.

This statement is universal again because it has the "for all", and existential because of the "there exists".

\subsection{Existential Universal Statement}

Ex. There exists a positive integer that is less than all the positive integers.

See above reasoning.

\section{Set Builder Notation}

A set is simply defined as a collection of objects. The size of a set is the number of unique elements in the set.

\subsection{Set Roster Notation}

\begin{itemize}
    \item Example set: $A = \{a, b, c\}$
    \item As shown, sets can include anything.
\end{itemize}

\subsection{Set Builder Notation}

\begin{itemize}
    \item Example set: $A = \{ x \in \mathbb{R} \mid -1 \le  x < 5 \}$
    \item $0 \in A$, $-2 \not\in A$
    \item Empty Set: $\{\}, \phi$
    \item Universal Set: $u$, set containing all objects, all other sets are proper subsets.
    \item Subsets:  $A \subseteq B$ if for any element $a \in A$, $a \in B$
    \item By definition, $\phi \subseteq A$
    \item Proper subset: $A \subset B$ means all elements in A are also in B, and there exist elements in B that do not exist in A.
    \item $\forall $: for all, $\exists $: there exists
\end{itemize}

We actually cannot use set roster notation most of the time. It is not countable.

\subsection{Cartesian Products}

\begin{itemize}
    \item $A \times B = \{(x, y)  \mid x \in A, y \in B\}$
    \item Number of elements in $A \times B$ = sizes multiplied together
    \item This was introduced in an effort to introduce ordered pairs (differentiation of $(a, b)$ and $(b, a)$ )
    \item Because of this, $A \times B \neq B \times A$
\end{itemize}

\section{The Language of Relations and Functions}

\begin{enumerate}
    \item Relations on sets
        \begin{itemize}
            \item Definition: a relation from set $A$ to set $B$ is a subset of $A \times B$. More formally, for some relation $x$, $x \subseteq A \times B$
            \item We denote a relation between $x_1$ and $x_2$ as $x_1 \: R \: x_2$ provided that $x$ is in  $A \times B$.
            \item It can be proved intuitively that a set with $n$ elements has  $2^n$ subsets.
            \item Domain and co domain of $A \times B$ is simply $A$ and  $B$ respectively.
        \end{itemize}
    \item Function from A to B
        \begin{itemize}
            \item Definition: a function is a relation such that $\forall x \in A, \exists y \in B  \mid (x, y) \in F$ where $F: A \mapsto B$
            \item If $(x, y) \in F$ and $(x, z) \in F$ then $(y, z) \in F$.
            \item In other words, every element in $A$ must have exactly one distinct image.
        \end{itemize}
    \item Extra Function Terminology
        \begin{itemize}
            \item Squaring function: $x \mapsto x^2$
            \item Successor function: $x \mapsto x + 1$
            \item Constant function: $x \mapsto C$
        \end{itemize}
\end{enumerate}

\chapter{The Logic of Compound Statements}

\section{Logical Form and Logical Equivalence}

An argument is a sequence of statements aimed at demonstrating the truth of an assertion.

\begin{itemize}
    \item The assertion at the end of the sequence is called the conclusion, and the preceding statements are called premises.
    \item We commonly use the variables $p$ $q$ and $r$ to represent component sentences.
\end{itemize}

A statement (or proposition) is a sentence that is true or false but not both.

Further terminology:

\begin{itemize}
    \item The symbol $\lsim$ denotes not, $\land$ denotes and, and  $\lor$ denotes or.
    \item And, or, and not can easily be represented using truth tables.
    \item Set up the truth table such that each group of columns builds off of the last.
    \item Two statement forms are logically equivalent iff they have identical truth table outputs. This is represented as $P \equiv Q$.
    \item Tautology \textbf{t}: A statement form that is always true regardless of the truth values substituted
    \item Contradiction \textbf{c}: opposite of tautology
    \item $p \land \textbf{t} \equiv p$,  $p \lor \textbf{c} \equiv p$.
    \item Absorption laws: $p \lor (p \land q) \equiv p$, $p \land (p \lor q) \equiv p$
    \item See page 35 of the Discrete Math Textbook for a complete table on logical equivalences.
\end{itemize}

De Morgan's Laws: $\lsim(p \land q) \equiv \lsim p \lor \lsim q$,  $\lsim (p \lor q) \equiv \lsim p \land \lsim q$.

Caution: De Morgan's Laws can only be used between complete statements on each side.

\section{Conditional Statements}

Logic flows from a hypothesis to a conclusion. The aim is to frame it as an "if, then". Given hypothesis $p$ and conclusion $q$, we represent this as \[
p \rightarrow q
.\] 

The only combination of circumstances in which you would call a conditional sentence false occurs when the hypothesis is true and the conclusion is false. If the statement is true because the hypothesis is false, this is called vacuously true.

Note that in order of operations, $\to $ is performed last, and also note that \[
p \to q \equiv p \lor \lsim q
.\].

Helpful Information:
\begin{itemize}
    \item The negation of "if p then q" is logically equivalent to "p and not q".
    \item $p \to q \equiv \lsim q \to \lsim p$.
    \item While the converse is not equivalent to the statement, the converse is logically equivalent to the inverse.
    \item $p$ \textbf{only if} $q$ means "if $p$ then $q$".
    \item Biconditional: "if and only if" is true if both statements have the same value.
\end{itemize}

\subsection{Necessary and Sufficient Conditions}

\begin{itemize}
    \item $r$ is a sufficient condition for $s$ means "if $r$ then $s$"
    \item $r$ is a necessary condition for $s$ means "if not $r$ then not $s$"
\end{itemize}

\section{Valid and Invalid Arguments}

\subsection{Arguments Terminology}

\begin{itemize}
    \item \textbf{Argument}: sequence of statements
    \item All statements in an argument except for the final one are called premises
    \item The final statement is called a conclusion.
    \item An argument form being valid means that if the resulting premises are all true, the conclusion is true.
    \item \textbf{Critical row}: role of the truth table in which all premises are true. If the conclusion in every critical row is true, the argument form is valid.
    \item \textbf{Syllogism}: argument form consisting of two premises and a conclusion. The first and second premises in a syllogism are the major and minor premises respectively.
    \item \textbf{Modus ponens}: If $p$ then $q$. $p$. $\therefore q$. Modus ponens means "method of affirming" in Latin.
    \item \textbf{Modus tollens}: If $p$ then $q$. $\lsim q, \therefore \lsim p$. Modus tollens means "method of denying" in Latin.

\end{itemize}

\subsection{Rule of Inference}

Rule of Inference: form of argument that is valid. Below are some helpful Rules of Inference.
\begin{itemize}
    \item \textbf{Generalization}: $p \therefore p \lor q$
    \item \textbf{Specialization}: $p \land q \therefore p$
    \item \textbf{Elimination}: $p \lor q, \lsim q \therefore p$
    \item \textbf{Transitivity}: $p \to q, q \to r \therefore p \to r$
    \item \textbf{Proof by casework}: $p \lor q, p \to r, q \to r \therefore r$
\end{itemize}

\subsection{Fallacies}

\begin{itemize}
    \item Using ambiguous premises
    \item Assuming that is to be proved
    \item Jumping to a conclusion
    \item Assuming the converse to be true.
    \item Assuming the inverse to be true.
    \item An argument is sound if and only if it is valid and all its premises are true.
\end{itemize}

\subsection{Contradiction Rule}

$\lsim p \to c \therefore p$, in other words, if you can prove that an assumption leads to a contradiction, then you have proved that the assumption is false.

\chapter{The Logic of Quantified Statements}

\section{Predicates and Quantified Statements I}

\subsection{Terminology}

\begin{itemize}
    \item \textbf{Predicate Calculus}: Symbolic analysis of predicates and quantified statements
    \item \textbf{Statement Calculus}: Symbolic analysis of ordinary compound statements (see Sections 2.1-2.3)
    \item \textbf{Predicate}: part of the sentence from which the subject has been removed. A predicate is a predicate symbol together with predicate variables. 
        Predicates become statements when specific values are substituted for the variables.
        \begin{itemize}
            \item Note that predicates can be obtained by removing some or all the nouns from a statement.
        \end{itemize}
    \item \textbf{Predicate symbols}: variables to stand for predicates that act as functions that take in predicate variables
    \item \textbf{Domain of a predicate variable}: set of all values that may be substituted in place of the variable.
    \item \textbf{Truth set of $P(x)$}: set of all elements of $D$ that make $P(x)$ true when they are substituted for $x$.
        \[
            \{x \in D  \mid P(x)\}
        .\] 
\end{itemize}

\subsection{The Universal Quantifier: $\forall $}

\begin{definition}{Universal Statement}{label}
    Let $Q(x)$ be a predicate and $D$ the domain of $x$. A \textbf{universal statement} (statement
    of the form $\forall x \in D, Q(x)$) is true if, and only if, $Q(x)$ is true for every $x$ in $D$. 
    It is false if there is a counterexample.
\end{definition}
One way to show that a universal statement is true is by showing that there are no counterexamples.
This is called the \textbf{method of exhaustion}.

\subsection{The Existential Quantifier: $\exists $}

\begin{definition}{Existential Statement}{label}
    Let $Q(x)$ be a predicate and $D$ the domain of $x$. An \textbf{existential statement} (statement
    of the form $\exists x \in D, Q(x)$) is true if, and only if, $Q(x)$ is true for at least one $x$ in $D$. 
    It is false if and only if $Q(x)$ is false for all $x$ in $D$.
\end{definition}

\subsection{Equivalent Forms of Universal and Existential Statements}

Observe that ($\forall x \in U$, if $P(x)$ then $Q(x)$) can always be rewritten as
$\forall x \in D, Q(x)$ by narrowing $U$ to be the domain $D$ where  $D$ consists of all values
$x$ that make $P(x)$ true.

\subsection{Implicit Quantification}

\begin{definition}{Implication Notation}{label}
    Let $P(x)$ and $Q(x)$ be predicates.
    \begin{itemize}
        \item $P(x) \implies Q(x) \equiv \forall x, P(x) \to Q(x)$, meaning every element in the truth
            set of $P(x)$ is also in the truth set of $Q(x)$.
        \item $P(x) \Leftrightarrow Q(x) \equiv \forall x, P(x) \leftrightarrow Q(x)$, meaning the two
            predicates have identical truth sets.
    \end{itemize}
\end{definition}

\section{Predicates and Quantified Statements II}

This section contains rules for negating quantified statements and additional extensions to quantified statements.

\begin{theorem}{Negation of a Universal Statement}{label}
    \[
    \lsim (\forall x \in D, Q(x)) \equiv \exists x \in D  \mid \lsim Q(x)
    .\] 
    In other words, the negation of a universal statement is that there exists a counterexample.
\end{theorem}

\begin{theorem}{Negation of an Existential Statement}{label}
    \[
        \lsim (\exists x \in D  \mid Q(x)) \equiv \forall x \in D, \lsim Q(x)
    .\] 
    In other words, the negation of an existential statement is the universal statement "none are" or "all are not".
\end{theorem}

Note that using the Negation of a Universal Statement theorem, we can find the negation of a universal conditional statement by
negating the quantifier and the conditional "separately".

\[
    \lsim (\forall x, P(x) \to Q(x)) \equiv \exists x  \mid P(x) \land ~Q(x)
.\] 

\subsection{Relation among $\forall ,\exists ,\land, \lor$}

Note that \[
    \forall x \in D, Q(x) \equiv Q(x_1) \land Q(x_2) \land \ldots \land Q(x_n)
.\] 
Similarly, \[
    \exists x \in D  \mid Q(x) \equiv Q(x_1) \lor Q(x_2) \lor \ldots \lor Q(x_n)
.\] 
Using this, we can easily prove the above two theorems using DeMorgan's Laws.
Additionally, note that contrapositives, converses, and inverses extend to universal conditional statements as well, and there is no need
to flip the quantifier when finding these.

\subsection{Necessary and Sufficient Conditions}

\begin{definition}{Sufficient/Necessary Conditions}{label}
    \begin{itemize}
        \item "$\forall x, r(x)$ is a \textbf{sufficient condition} for $s(x)$" means "$\forall x, r(x) \to s(x)$".
        \item "$\forall x, r(x)$ is a \textbf{necessary condition} for $s(x)$" means "$\forall x, \lsim r(x) \to \lsim s(x)$".
    \end{itemize}
\end{definition}

\section{Statements with Multiple Quantifiers}

When there are multiple quantifiers, we perform the quantifiers in the order that they are stated. In terms of coding, it may be helpful to think
of the first quantifier as the outer loop and the second quantifier as the inner loop in a case with 2 quantifiers.
Amazingly, note that the order of the quantifiers only matters between $\forall $ and $\exists $.

\subsection{Negations of Multiply-Quantified Statements}

We simply negate the statement from left to right.

\begin{align}
    \lsim (\forall x \in D, \exists y \in E  \mid P(x, y)) \\
    \equiv \exists x \in D  \mid \lsim (\exists y \in E  \mid P(x, y)) \\
    \equiv \exists x \in D  \mid \forall y \in E, \lsim P(x, y)
\end{align}

\section{Arguments with Quantified Statements}

\begin{definition}{Rule of Universal Instantiation}{label}
    If some property is true of \emph{everything} in a set, then it is true of \emph{any particular} thing in the set.
\end{definition}

\subsection{Universal Modus Ponens/Tollens}

Rule of Universal Instantiation can be used in combination with Modus Ponens in order to establish that a universal conditional statement can establish
that the necessary condition necessarily follows from the sufficient condition.
Forms of Modus Ponens and Modus Tollens can be found in Section 2.3.1.

\subsection{Disc Diagrams}

They're basically inheritance diagrams. Since I don't know how to draw with LaTeX yet, just check 3.4.5 of the book for more info.

\subsection{Universal Transitivity}

\begin{definition}{Universal Transitivity}{label}
    If $\forall x P(x) \to Q(x), Q(x) \to R(x)$, then $\forall x P(x) \to R(x)$.
\end{definition}

\end{document}

